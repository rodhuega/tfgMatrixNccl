%!TEX root = ../tfgRodhuega.tex

\chapter{Introducción general\label{capIntroduccion}}

En este primer capítulo de presentación de la memoria, se 
introduce al lector en la temática de la tesis de máster, empezando con la sección de 
motivación, en la que se des\-cri\-be el interés del tema. En el segundo 
apartado, estado del arte, se muestra el contexto científico-técnico 
en el que se encuentran los avances en materia relacionada con el ahorro de 
energía en álgebra lineal densa. Finalmente, se presenta los ob\-je\-ti\-vos 
específicos del trabajo.

\section{Motivación}

Desde décadas, la computación de altas prestaciones ha concentrado sus 
esfuerzos en la optimización de algoritmos aplicados a la resolución de 
problemas complejos que aparecen en un amplio abanico de aplicaciones de casi 
todas las áreas científicas y tecnológicas. En particular, problemas de sistemas 
de ecuaciones lineales o problemas de mínimos cuadrados aparecen frecuentemente 
durante el análisis y el estudio del campo gravitatorio terrestre, la simulación 
del comportamiento de componentes estructurales en aviación o en la detección de 
enfermedades a partir de resonancias magnéticas. Para todos estos casos, la 
resolución de estos problemas supone la parte computacionalmente más costosa 
para la obtención de resultados.

Por esa razón, el principal objetivo de la computación de 
altas prestaciones es la optimización mediante en el uso de herramientas y 
técnicas, tales como la computación paralela, de problemas en ingeniería. En 
este contexto, el término optimización hace referencia a la reducción 
tiempo de ejecución, aunque también  al espacio necesario para su cómputo. La 
mayoría de estos logros han sido posibles gracias al avance tecnológico de los 
componentes de los computadores impulsado principalmente por los fabricantes de 
hardware. Concretamente, las unidades centrales de procesamiento (CPU) o 
procesadores vienen doblando la velocidad y el número de transistores cada 18-24 
meses en las últimas décadas. Desde 200, la triple barrera del consumo energético,
las limitaciones en el paralelismo de instrucción y la elevada latencia a memoria,
ha provocado que el diseño de procesadores multinúcleo se haya convertido en la 
única vía para transformar el creciente número de transistores en un aumento
del rendimiento.

En este sentido, el incremento de la velocidad de los procesadores así como el aumento 
del número de los núcleos en éstos ha sido un factor clave para el conseguir 
mejores prestaciones. No obstante, el aumento de la frecuencia de los 
procesadores junto con el uso de la tecnología multinúcleo ha implicado que el 
consumo energético necesario para su funcionamiento también haya crecido y se 
haya convertido, hoy en día, en un factor muy importante a tener en cuenta. La búsqueda de 
soluciones verdes o fuentes de energía alternativas que permitan reducir las 
emisiones de CO$_2$ a la atmósfera demuestran la creciente preocupación por 
el medio ambiente. En el ámbito de las tecnologías de la información y, más 
concretamente, en la computación de altas prestaciones, la comunidad 
cientifico-técnica actualmente muestra especial interés en el desarrollo de 
componentes, herramientas y técnicas que permitan minimizar el consumo 
energético. Medidas como FLOPS/Watt~\cite{flopswatt}, \emph{Energy-To-Solution}~\cite{Minartz} 
o FTTSE~\cite{Bekas} están empezando a tomar importancia cuando se evalúan 
las prestaciones de algoritmos y computadores: de hecho, se ha creado un \emph{ranking}
como el Green500 \cite{feng2009green500}, análogo al Top500, que ya utiliza este tipo de métricas para comparar 
y clasificar los supercomputadores en el mundo.

Algunas herramientas de ahorro, basadas en la transición de la computadora a estados 
de bajo consumo o en la reducción de la frecuencia y el voltaje de forma 
dinámica (DVFS) en los procesadores, ofrecen la posibilidad de generar aplicaciones conscientes
del consumo. Por ejemplo, sistemas autónomos de ahorro energético para 
grandes plataformas (clusters de computadores y supercomputadores) y en máquinas 
de sobremesa están empezándose a implantar con el objetivo de limitar el consumo y reducir 
los costes económicos generados tanto por las propias máquinas como por los sistemas 
de refrigeración.

En resumen, los algoritmos paralelos de álgebra lineal densa
aplicados a problemas de ingeniería y las técnicas de ahorro de energía 
disponibles en procesadores mutinúcleo actuales son las dos vertientes que se 
aúnan en este trabajo. Concretamente, el objetivo del trabajo es realizar 
un estudio preliminar de un \emph{planificador consciente del consumo en 
algoritmos de álgebra lineal densa sobre procesadores multinúcleo} mediante
el uso de DVFS. 

\section{Estado del arte}

La temática sobre el ahorro de energía está tomando, cada vez más, especial 
relevancia en la computación de altas prestaciones. Nuevas técnicas, 
herramientas, componentes y multitud de algoritmos, intentan de algún modo, 
reducir la energía consumida. En este sentido los problemas de álgebra lineal 
densa, necesitan, cada vez más, de potentes plataformas para aumentar su 
rendimiento y prestaciones. Estas plataformas, compuestas por una gran cantidad
procesadores multinúcleo que operan a altas frecuencias, provocan grandes consumos 
energéticos lo que las convierte en opciones no deseadas a nivel 
económico en muchos casos. Por esta razón, centros de investigación, 
universidades y empresas dedican gran parte de sus esfuerzos a buscar 
nuevas soluciones y alternativas para desarrollar aplicaciones conscientes
del consumo energético. En esta sección, se explican algunas de las 
técnicas, métodos y algoritmos utilizados en planificadores, y en segundo lugar,
la técnica de reducción de energía disponible en procesadores, basada en el
escalado dinámico de la frecuencia y el voltaje (DVFS).

\subsection{Planificadores}

El estudio de tareas paralelas y distribuidas se ha abordado en detalle con
el objetivo de que los procesadores optimicen el tiempo y respeten las dependencias
entre tareas cuando se ejecuta una aplicación.

Normalmente los algoritmos de planificación suelen clasificarse en dos categorías:
estáticos y dinámicos. En los estáticos, la planificación y la asignación de 
recursos a las tareas se realiza antes de que las aplicaciones se ejecuten, realizándose
el supuesto de partida de que se conoce a priori el coste de cada una de las tareas y las 
comunicaciones entre ellas. Además, asumen que las tareas ocupan el procesador
asignado hasta que finalizan \cite{Lo:1988:HAT:52029.52035,Sarkar:1987:PSP:37581}. 
Por otra parte, los algoritmos dinámicos planifican las tareas en tiempo de ejecución, 
aplicando técnicas de equilibrado de la carga. 

Los planificadores por listas son los planificadores estáticos más conocidos. 
Éstos poseen listas donde se ubican las tareas a ser ejecutadas, ordenadas 
a través de ciertas prioridades \cite{Rongheng,Mtibaa}. En este trabajo 
se abordará un planificador estático que aplicará una determinada política, 
donde se conoce a priori la duración de las tareas obtenidas a partir
del orden de coste teórico de las mismas.

\subsection{Ahorro de energía: escalado dinámico de la frecuencia y el voltaje}

El escalado dinámico de la frecuencia y el voltaje se ha convertido, hoy en día,
en una característica prácticamente presente en todas de las nuevas generaciones de 
procesadores multinúcleo \cite{hsu2005feasibility,hsu2005power}. La reducción de la frecuencia de reloj
del procesador, y la consecuente reducción del voltaje necesario durante periodos
ociosos o de baja demanda, dan como resultado final una importante reducción del
consumo requerido. No obstante, hay que ser conscientes de que la reducción
de la frecuencia tiene asociada un aumento de los tiempos de ejecución.
En \cite{LaszewskiWYH09} se definen los clusters DVFS, que son capaces de reducir la
frecuencia del reloj en periodos de baja actividad. Actualmente existen numerosas
técnicas de DVFS que pueden aplicarse en un amplio abanico de posibilidades,
dentro del marco de la computación de altas prestaciones. Por ejemplo, en 
grandes centros de datos de alta producción y disponibilidad, para reducir el 
consumo del conjunto en total \cite{DBLP:journals/computer/GortonGSW08,feng2007green}. 

En este ámbito, existen diferentes métodos que emplean como herramienta de ahorro 
de energía el escalado dinámico de la frecuencia y el voltaje, tales como:

\begin{itemize}
 \item Análisis del grafo de dependencias (\emph{Acyclic Directed Graph}, DAG) de 
aplicaciones científicas, donde se identifica el camino crítico, siendo posible
reducir el consumo de aquellas tareas no críticas \cite{Chen:2005:RPP:1053738.1054680}.

 \item Otras aplicaciones \cite{ge2007cpu} dedicadas a trabajar en conjunto con el planificador
del sistema operativo, para ajustar de forma dinámica en tiempo real la frecuencia
de los procesadores.

 \item Técnicas de reducción de la frecuencia en aplicaciones paralelas durante
los periodos de comunicación, como por ejemplo MPI \cite{freeh2005using,roundtree2007adaptive}.

 \item Además de las aplicaciones paralelas, los planificadores de máquinas
virtuales también tienen la posibilidad de utilizar DVFS \cite{LaszewskiWYH09}.
\end{itemize}

En este trabajo se emplea DVFS con el propósito de realizar una planificación eficiente,
a partir del grafo dirigido de dependencias, sobre aplicaciones y algoritmos asociados al álgebra lineal densa.

\subsection{Planificación consciente del consumo}

Si se busca el trabajo desarrollado relacionado con la planificación y el
empleo de DVFS como herramienta de reducción de consumo, podremos observar que 
existen numerosas investigaciones y artículos sobre esta temática. Por ejemplo, en 
\cite{Etinski} se modela un planificador para clusters capaz de asignar 
tareas en tiempo real y capaz de regular la frecuencia de  
los procesadores en función de la carga de trabajos en un determinado momento; es decir, 
asignando tareas en periodos de no utilización y constrastando, finalmente los ahorros energéticos 
producidos.

En \cite{Yao,Manzak} se discute cómo planificar tareas independientes con DVFS en un 
monoprocesador; en \cite{yeon, Gruian} se emplea DVFS para planificar tareas
con dependencias en múltiples procesadores; y en \cite{Martin,Luo,Luo2} se citan
algunos algoritmos de planificación a tiempo real para tareas dependientes. En 
\cite{Zhang} se presenta una plataforma que integra la asignación de tareas a tiempo real, 
utilizando DVFS para minimizar el consumo en tareas con dependencias obteniendo resultados sobre
problemas de programación lineal entera. LPHM \cite{Robert} es un 
planificador dinámico que intenta maximizar el tiempo de las tareas no 
críticas mediante DVFS.

En \cite{lee2009minimizing} se proponen heurísticas para un planificador consciente del consumo
de tareas paralelas en entornos de clusters heterogéneos. En \cite{kimura2007emprical} se
emplea una estrategia, aplicada sobre clusters, basada en la reducción de las holguras
de las tareas no críticas.

La intención de este trabajo es realizar una planificación consciente del consumo
a través del uso de DVFS, basada en la misma idea que se comenta en párrafos anteriores. 
La reducción de las holguras entre tareas no críticas del grafo de dependencias puede
conseguir importantes ahorros de energía, sin perjudicar el rendimiento de los algoritmos
ejecutados.  El objetivo, por tanto, es determinar hasta qué punto una tarea no crítica
puede ralentizarse y plasmar el resultado de esta investigación en forma de un \emph{algoritmo 
de extensión de tiempos de tareas}. Como segundo objetivo, el algoritmo de extensión de tareas que se 
diseña se evaluará mediante un simulador, desarrollado también en el marco de esta tesis.
En nuestro caso, el estudio hará especial hincapié en algoritmos comúnmente utilizados 
en álgebra lineal densa, tales como la descomposición en factores de la matriz densa ligada 
a un sistema de ecuaciones lineales a través de métodos de Cholesky, LU, QR o LDL$^T$.


\section{Objetivos}

La intención de nuestro trabajo es realizar una planificación consciente del consumo
a través del uso de DVFS en los procesadores donde se ubiquen las tareas no 
críticas del grafo de dependencias entre tareas que representa las tareas en las 
que se subdivide un algoritmo paralelo de álgebra lineal densa. Por lo tanto, los objetivos concretos del trabajo 
son los siguientes:

\begin{itemize}

\item Búsqueda de información, técnicas y métodos e investigación sobre trabajos relacionados
que puedan servir como referencia para nuestro objetivo. Palabras clave para la búsqueda: 
algoritmos de álgebra lineal densa, técnicas de ahorro de energía en procesadores multinúcleo, teoría de grafos y
métodos de administración y planificación de tareas.

\item Diseño y elección de un sistema de representación de grafos flexible, cómodo y fácil
de tratar y visualizar. La elección de este sistema repercutirá en un futuro en la forma
de manejar las dependencias entre tareas, por lo que la decisión tomada repercutirá
en la implementación realizada del algoritmo de extensión de holguras y simulador.

\item Búsqueda de algoritmos de álgebra lineal a bloques que puedan descomponerse
en subtareas. Al mismo tiempo se intentarán elaborar métodos de generación automática
de grafos a partir de algoritmos básicos. Este banco de pruebas permitirá, una vez implementado el
planificador consciente del consumo, evaluar su rendimiento.

\item Diseño e implementación de un algoritmo que permita, a través del grafo de dependencias de
un algoritmo de álgebra lineal densa, analizarlo, detectar las dependencias entre
tareas y determinar para cuales de ellas puede extenderse su duración en función de un
rango de frecuencias discreto como parámetro de entrada. Este método
deberá devolver, en forma de grafo, la frecuencia mínima a la que debe ejecutarse cada tarea 
para que el algoritmo de entrada no pierda prestaciones cuando sea ejecutado.

\item Diseño e implementación de un planificador a modo de simulador que permita
planificar las tareas del algoritmo anterior, permitiendo diferentes modos de planificación
y configuración del número de procesadores multinúcleo que se emplearán. Finalmente
esta traza de simulación deberá devolver estadísticas y porcentajes de tiempo, para cada 
procesador, a los que ha estado trabajando en cada frecuencia. Estos resultados
servirán para verificar y evaluar el comportamiento de la herramienta implementada.

\item Evaluación del planificador implementado mediante diferentes
algoritmos de álgebra lineal densa. Principalmente se realizarán experimentos
con algoritmos por bloques de descomposiciones de matrices cuadradas y densas. Estos algoritmos
son:

\begin{itemize}
\item Cholesky por bloques.
\item QR por bloques.
\item QR por bloques de columnas (tareas del mismo tipo con diferente coste).
\end{itemize}

\item Conclusiones sobre los resultados de ahorro obtenidos con los algoritmos de álgebra 
lineal densa escogidos y discusión de trabajos futuros relacionados, que darán pie a la 
tesis doctoral en esta linea de investigación.

\end{itemize}

Esta tesis de máster, como ya se ha comentado, servirá como base para futuros trabajos orientados
a las líneas de computación de altas prestaciones y al ahorro de energía sobre
procesadores multinúcleo. Al mismo tiempo, la tesis introducirá la temática de
la tesis doctoral que se pretende llevar a cabo en los próximos años.

\subsection{Limitaciones del trabajo\label{limitaciones}}

A primera vista, la implementación de un simulador de planificación de tareas
consciente del consumo parece sencilla. Sin embargo, basta consultar en la temática y teoría
de planificación de tareas para comprobar que una planificación óptima de tareas, salvo en condiciones muy particulares,
es un problema NP-completo. Si además se añade la posibilidad de que cada una de 
estas tareas puede ejecutarse a diferentes frecuencias y, por lo tanto,
cambiar su duración, la complejidad del trabajo aumenta aún más. 

En esta tesis, al abordarse una aproximación teórica, que servirá como base hacia 
la futura tesis doctoral, se ha decidido acotar el estudio a una serie de casos particulares. 
Estas limitaciones son las siguientes:

\begin{itemize}
 \item Uno de los objetivos principales de este trabajo es implantar un planificador
consciente del consumo en algún runtime de librerías de computación numérica, como
libflame o SuperMatrix. En este trabajo sólo se implementará un planificador a modo
de simulador que permita obtener resultados y estadísticas del ahorro producido, 
sin llegar a se incorporado en núcleos de estas librerías.

 \item El coste de realizar un cambio de frecuencia del procesador aplicando DVFS
no es despreciable sino que depende del cambio de frecuencia realizado, es decir, de la frecuencia actual 
y la frecuencia destino. En este trabajo se asume un coste constante de cambio de 
frecuencia (actual y destino) para el rango de frecuencias de procesador aceptadas.

 \item Cuando se cambia la frecuencia del procesador, se asume que la frecuencia
únicamente puede cambiarse a nivel de procesador o socket y no a nivel de núcleo.
Por este motivo, cuando se cambie la frecuencia en un procesador, todos los núcleos
asociados a éste adoptarán la nueva frecuencia después del tiempo de cambio de frecuencia.

 \item El algoritmo de expansión de tiempos de tareas asume que los recursos
en la plataforma donde se simulará la planificación son infinitos.

 \item Con el objetivo de limitar el número de combinaciones posibles de planificación
se asume que únicamente se puede cambiar la frecuencia de procesador
cuando todos los núcleos del mismo están libres. Es decir, no se podrá
cambiar la frecuencia del procesador durante la ejecución de una tarea en cualquiera
de los núcleos.

 \item En una situación de simulación real, se recogen datos de tiempo realísticos de la
ejecución de las tareas en cada una de las frecuencias permitidas por el procesador.
En este trabajo se asume que el coste temporal del las tareas crece de forma inversamente
proporcional a la frecuencia del procesador.
\end{itemize}

\newpage
\section{Organización de la memoria}

La memoria de la tesis de máster se estructura en los siguientes capítulos:

\begin{itemize}

\item \textbf{Aproximación teórica}

En este Capítulo~\ref{capPlanificacion} se revisan los fundamentos teóricos de 
algunos de los métodos y técnicas de revisión y evaluación 
de programas aplicados a la planificación de proyectos de 
ingeniería. A continuación, se evalúa el uso de esta técnica
en algoritmos de álgebra lineal densa. Finalmente se definen 
una serie de restricciones y simplificaciones que se adoptan  para limitar 
el número de casos posibles de estudio en esta tesis de máster.

\item \textbf{Implementación}

En el Capítulo~\ref{capDescripcion} se describen los módulos y el planificador implementado 
para simular las ejecuciones de las tareas a partir del algoritmo de
ajuste de holguras explicado en el capítulo anterior. A continuación,
se explicarán las políticas que se han tenido en cuenta
y se probará, con ejemplos a modo explicativo, para analizar cómo se ejecuta la simulación. 

\item \textbf{Evaluación de algoritmos}

En el Capítulo de~\ref{capEvaluacion} se evaluarán, mediante los
grafos de dependencias asociados, diversos algoritmos de álgebra lineal
densa, tales como Cholesky, QR y QR por bloques de columnas.
Al mismo tiempo se presentan las conclusiones sobre los ahorros
de energía producidos y, a través de estadísticas, se demuestra
la factibilidad del algoritmo de ajuste de holguras implementado.

\item \textbf{Conclusiones}

En el Capítulo~\ref{capConclusiones} se hace una recopilación de las
ideas destacadas, tratando de enfatizar los aspectos más interesantes, 
los conceptos aprendidos y la aportación en la temática de ahorro
de energía al álgebra lineal densa.

\item \textbf{Futuras extensiones}

En el Capítulo~\ref{capTrabajoFuturo} se exponen las posibles mejoras o
extensiones que pueden aplicarse al simulador. En particular, se
comentan las intenciones y las posibles aplicaciones en un futuro
no lejano de la viabilidad en la aplicación de esta técnica en
librerías de cálculo de altas prestaciones.

\end{itemize}

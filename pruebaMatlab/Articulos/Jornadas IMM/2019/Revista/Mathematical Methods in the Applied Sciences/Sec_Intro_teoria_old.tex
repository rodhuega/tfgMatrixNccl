\section{Introduction and notation}

The computation of matrix trigonometric functions has received remarkable
attention in the last decades due to its usefulness in the solution of systems of
second order linear differential equations. Recently, several state-of-the-art
algorithms have been provided for computing these matrix functions, see \cite{Serb80, dehghan2010computing, High08, alonso2018computing}, in
particular for the matrix cosine function.\\

%The proposed methods for calculating the matrix cosine can be classified into two classes: \emph{Rational approximation methods} and \emph{Polynomial methods}.\\


Among the proposed methods for the approximate computation of the matrix cosine, two fundamental ones stand out: those based on rational approximations \cite{tsitouras2014bounds, Serb79, Serb80, AlHR15}, and those related to  polynomial approximations, using either Taylor series developments  \cite{sastre2017two, sastre2019fast} or serial developments of Hermite matrix polynomials  \cite{defez2019efficient}. In general, polynomial approximations showed to be more efficient than the rational algorithms in tests because they are more accurate despite a slightly higher cost.\\


Bernoulli polynomials and Bernoulli numbers have been extensively used in several areas of mathematics (an excellent survey about Bernoulli polynomials and its applications can be found in  \cite{kouba2013lecture}).\\
% The development of series functions of Bernoulli polynomials has been studied in \cite{costabile2001expansion,costabile2001expansions}.\\


In this paper, we will present a new series development of the matrix cosine in terms of the Bernoulli matrix polynomials. We are going to verify that its use allows obtaining a new and competitive method for the approximation of the matrix cosine. \\



The organization of the paper is as follows: In section \ref{section2}, we will obtain two serial developments of the matrix cosine in terms of the Bernoulli matrix polynomials. In section \ref{section3}, we will present the different numerical tests performed. Conclusions are given in section \ref{section4}.\\


 Throughout this paper, we denote by $\mathbb{C}^{r \times r}$ the set of all the complex square matrices of size $r$. Besides, we denote $I$ as the identity matrix in $ \mathbb{C}^{r \times r}$. A polynomial of degree $m$ is given by an expression of the form $P_m(t)=a_{m} t^m+a_{m-1}t^{m-1}+\cdots+a_{1}t+a_{0}$, where $t$ is a real variable and $a_j$, for $0\leq j \leq m$, are complex numbers. Moreover, we can define the matrix polynomial $P_m(B)$ for $B \in \mathbb{C}^{r \times r}$  as $P_m(B)=a_{m} B^m+a_{m-1}B^{m-1}+\cdots+a_{1}B+a_{0}I$.  As usual, the matrix norm $\left\|\cdots \right\|$ denotes any subordinate matrix norm; in particular $\left\| \cdots \right\|_{1}$ is the usual $1-$norm.

\section{On Bernoulli matrix polynomials}\label{section2}
The Bernoulli polynomials $B_n(x)$ are defined in \cite[p.588]{olver2010nist} as the coefficients of the generating function

\begin{equation}\label{Bernoulli1}
g(x, t)= \frac{t e^{tx}}{e^t-1}=\sum_{n \geq 0} \frac{B_n(x)}{n!}t^n  \ , \ |t|<2\pi,
\end{equation}
where $g(x, t)$ is an holomorphic function in $\mathbb{C}$ for the variable $t$ (it has an avoidable singularity in $t=0$). Bernoulli polynomials  $B_n(x)$ has the explicit expression

\begin{equation}\label{Bernoulli2}
B_n(x)=\sum_{k=0}^{n} {n \choose k} B_k x^{n-k},
\end{equation}
where the Bernoulli numbers are defined by $B_n=B_n(0)$. Therefore, it follows that the Bernoulli numbers satisfy

\begin{equation}\label{Bernoulli3}
\frac{z}{e^z-1}=\sum_{n \geq 0} \frac{B_n}{n!}z^n  \ , \ |z|<2\pi,
\end{equation}
where
\begin{equation}\label{Bernoulli3a}
B_0=1, \displaystyle  B_{k}= -\sum_{i=0}^{k-1} {k \choose i} \frac{B_i}{k+1-i}, k \geq 1.
\end{equation}
Note that $ B_{3}=B_{5}=\cdots=B_{2k+1}=0$, for
$k\geq 1$. For a matrix $A \in \mathbb{C}^{r \times r}$, we define the $m-th$ Bernoulli matrix polynomial by the expression

\begin{equation}\label{Bernoulli2matrix}
B_m(A)=\sum_{k=0}^{m} {m \choose k} B_k A^{m-k}.
\end{equation}

We can use the series expansion

\begin{equation}\label{Bernoulli4}
e^{At} = \left(\frac{e^t-1}{t}\right)\sum_{n \geq 0} \frac{ B_n(A) t^{n}}{n!} \ , \ |t|<2\pi,
\end{equation}

to obtain approximations of the matrix exponential. A method based in (\ref{Bernoulli4}) to approximate the exponential matrix has been presented in \cite{defez2019}.\\

From (\ref{Bernoulli4}), we obtain the following expression for the matrix cosine and sine:

\begin{equation}\label{Bernoulli9buena}
\left.\begin{array}{rcl}
\cos{(A)} &=&\displaystyle  \left( \cos{(1)}-1\right)\sum_{n \geq 0} \frac{(-1)^n B_{2n+1}(A)}{(2n+1)!}+ \sin{(1)}\sum_{n \geq 0} \frac{(-1)^n B_{2n}(A)}{(2n)!}, \\
\\
\sin{(A)} &=& \displaystyle   \sin{(1)}\sum_{n \geq 0} \frac{ (-1)^n B_{2n+1}(A)}{(2n+1)!}-\left(\cos{(1)}-1\right)\sum_{n \geq 0} \frac{ (-1)^n B_{2n}(A)}{(2n)!}.
\end{array} \right\}
\end{equation}

Note that unlike the Taylor (and Hermite) polynomials that are even or odd, depending on the parity of the polynomial degree $n$, the Bernoulli polynomials do not verify this property. Thus,
in the development of $\cos{(A)}$ and $\sin{(A)}$, all Bernoulli polynomials are needed (and not just the even-numbered ones).\\

Replacing in (\ref{Bernoulli4}) the value $t$ for $it$ and $-it$ respectively and taking the arithmetic mean, we obtain the expression

\begin{equation}\label{Bernoulli11buena}
\sum_{n \geq 0} \frac{(-1)^n B_{2n}(A)}{(2n)!}t^{2n} =\frac{t}{2 \sin{\left( \frac{t}{2} \right)}}\left(\cos{\left(t A- \frac{t}{2}I \right)}  \right)  \ , \ |t|<2\pi.
\end{equation}


Taking $t=2$ in (\ref{Bernoulli11buena}) it follows that

\begin{equation}\label{Bernoulli10buena1}
\cos{(A)} = \sin{(1)}\sum_{n \geq 0} \frac{(-1)^n 2^{2n} B_{2n}\left(\frac{A+I}{2}\right)}{(2n)!},
\end{equation}
Note that in formula (\ref{Bernoulli10buena1}) only even grade Bernoulli's polynomials appear.
